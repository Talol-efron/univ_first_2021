\documentclass[a4paper, 11pt, titlepage]{jsarticle}
\usepackage{listings}
\usepackage[dvipdfm]{graphicx}

\title {知能情報実験1 : レポート課題1}
\author{205713B  朝比奈 太郎}
\date{\today }

\begin{document}
\maketitle
\tableofcontents
\clearpage

\section{実験の目的と内容}
一般的にコンパイラ言語、中間言語、インタプリタ言語の順番に実行速度が速いと言われているが実際のところ、コンパイラ言語とインタプリタ言語の実行速度にどの程度の差があるのか実験する。本実験では、いくつかの典型的な処理パターンを複数の言語で実装し、処理速度を比較して確認する。また、コンパイラ言語の代表としてC,インタプリタ言語の代表としてPython,両者の中間としてJavaを用いて実行速度の比較を行う.

\section{実験1の事前予想と結果と考察}
\subsection{事前予想}
今回の実験は「素数の数を調べる」というもので、for文とif文を用いる単純なコードであるため、一般的に言われる、コンパイラ言語、中間言語、インタプリタ言語の順番に実行速度が速いと考える。またC言語の最適化の有無は、最適化を行ったコンパイルの方が早いと考える。
従って、c with clang(最適化有り)-  c with clang(最適化無し)-  Java -  Python
の順番で実行速度が速いと予想する。
\subsection{結果}
\begin{itemize}
\item c with clang(最適化有り) : 0.265
\item c with clang(最適化無し) : 0.296
\item java : 0.556
\item Python : 13.652
\end{itemize}
\subsection{考察}
コードが単純なためか、C言語において、最適化ありとなしでは差が0.03秒で大きな差は見られなかった。javaは,コンパイル言語とインタプリタ言語の中間言語であるが、タイムはコンパイル言語にかなり近く(0.3秒差)、ほとんどコンパイラ言語と変わらない結果となった。Pythonは、13.652秒と他の3つの項目と比べて大きく実行速度の差が出た。これは、C, javaはともにユーザーが手動でコンパイルしてからの実行であるが、Pythonはコンパイルも一気にできてしまうという観点からコンパイル分の時間がかかっているのではないかと考える。また、JavaScriptなどの他言語には実行のたびにソースコードをコンパイルし、実行速度の高速化を測るJIT(Just in time)コンパイラーがあるが、Pythonにも存在しないわけではないが、既存の処理系を置き換えるほどの成功には至っていないことが挙げられる。

\section{実験2の事前予想と結果と考察}
\subsection{事前予想}
今回の実験は「ファイルアクセス」というもので、0以上10,000,000未満のすべての整数を昇順に文字列としてをファイルに書き込むコードであるため、ファイルの読み書きに強いまたはよく使われている感じのするPythonが一番実行速度が速いと予想。続いて、c(最適化あり), c(最適化なし),  java の順番。

\subsection{結果}
\begin{itemize}
\item c with clang(最適化有り) : 1.013
\item c with clang(最適化無し) : 1.014
\item java : 0.955
\item Python : 5.354
\end{itemize}
\subsection{考察}
ファイルアクセスを伴うコードの実行速度は,ディスクドライブ等のアクセス速度,OSの処理,など外部要因が(プログラマが描いたコードやコンパイラ)大きく影響される。
このことから、C言語の最適化の有りと無しで実行速度に変わりがなかったのは、他のファイルにアクセスし、書き込むという行為に最適化が適応されなかったためであると考える。
また、私は事前にpythonの実行速度が最もはやいと予想していたが、やはり自身でコンパイルをするという時間がどうしてもかかってしまい、処理速度が他の3つに比べて遅くなったと考える。

\section{実験3の事前予想と結果と考察}
\subsection{事前予想}
複雑な処理においては,処理系(言語)の違いよりも,選択するアルゴリズムの影響が大きくなることがある.スクリプト系の言語は,特に文字列,連想配列,ソートなどよく使われる処理の優れたライブラリを内蔵しており,より短いコードで効率的に動作する.
このことより、唯一のスクリプト言語であるPythonが効率的に動作し、一番の実行速度を誇ると予想。次に実験1、2よりC(最適化あり)が最も早い結果を出したので、C(最適化あり)が2番目。最後に、C(最適化なし)の順番になると予想する。
\subsection{結果}

\begin{itemize}
\item c with clang(最適化有り) : 1.965 
\item c with clang(最適化無し) : 3.397
\item java : xxx
\item Python : 0.223
\end{itemize}

\subsection{考察}
\section{使用するプログラミング言語の選定指針}
\subsection{本実験を通じて得られた結果から、言えること}
実験1,2,3において、1から3にかけて実験内容(コーディング)が複雑になっていた。それに伴い、コンパイラ言語であるC言語、中間言語であるjavaの2言語の実行速度は実験が複雑になるにつれて遅くなった。しかし、インタプリタ言語であるPythonは実験の複雑さに反比例して実行速度が速くなった。
したがって、単純な処理においては、基本的に処理速度の速いコンパイラ言語を用いるべきである。一方で、複雑な処理において、スクリプト系の言語は,特に文字列,連想配列,ソートなどよく使われる処理の優れたライブラリを内蔵しており,より短いコードで効率的に動作することからPythonといったインタプリタ言語を用いるべきである。

\subsection{一般論として、どの言語がどんなプロジェクトに利用されているのか}
PythonとCでは、可読性が高いpythonの方が開発が容易だといえる。また、C言語のようなコンパイル言語は、プログラムをまとめてコンパイルするため、エラーもまとめて対応しなければならない。つまり、Pythonのようなインタプリタ言語に比べてデバッグがしづらいというデメリットもあるため、Pythonの方が一般的に開発は容易だといえる。
どんなプロジェクトでもPythonをつかべきかというとそうではなく、実験1,2,3を通してでた単純な処理であればCの方が処理速度が早いという結果を元に単純なプロジェクトや、プログラムの実行速度が求められるプロジェクトの場合はCの方が適しているといえる。仮に、メンバのほとんどがjavaプログラマの場合であれば、javaを用いて処理するのが好ましいが、javaはcの派生言語であり、javaプログラマであればcを習得することが他のプログラマに比べて容易であるため、cを習得し、cでコーディングするのが良いと思う。
\begin{thebibliography}{99}
\bibitem{Xtech} ズバリPythonの弱点は?AI以外の有望分野は?気になる疑問に ..., https://xtech.nikkei.com/atcl/nxt/cpbook/18/00034/00003/, 2021/6/15.
\bibitem{tech-camp}  コンパイラ型言語とインタプリタ型言語のメリット, https://tech-camp.in/note/technology/93845/ , 2021/6/18
\bibitem{akamine} 赤嶺, 2021実験1-1プログラムの実行速度, 2021/6/18
\end{thebibliography}


\end{document}