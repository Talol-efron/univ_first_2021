\documentclass[11pt]{jsarticle}
\usepackage{amsmath}
\begin{document}
レポート4
\begin{flushright}
\today
\end{flushright}
\begin{flushright}
  205713B\ 朝比奈太郎
\end{flushright}
母分散が未知なので、検定統計量Tは自由度n-1のt分布に従う。
\[
T_0 = \frac{\bar X - \mu}{\sqrt{4/n}}
\]
帰無仮説$\mu=0$に対して有意水準5\%の片側検定なので、上側0.05点は、
\[
t(n-1, 0.05) = \frac{\bar x}{\sqrt{4/n}}
\]
検出力が0.9以上なので、対立仮説$\mu=1$に対して下側0.1点を考えれば良いから
\[
-t(n-1, 0.1)>= \frac{\bar x-1}{\sqrt{4/n}} = t(n-1, 0.05) - \frac{1}{\sqrt{4/n}}
\]
\[
\frac{t(n-1,0.05)+t(n-1,0.1)}{\sqrt{n}}>=1/2
\]
t分布と照らし合わせると、n=35のとき左辺は0.50673で、n=36のとき左辺は0.4993となり、36以上の標本数が必要であるといえる。
\end{document}
