\documentclass[11pt]{jsarticle}
\usepackage{amsmath}
\begin{document}
レポート3
\begin{flushright}
\today
\end{flushright}
\begin{flushright}
  205713B\ 朝比奈太郎
\end{flushright}
\section{}
母数$\mu $, $\sigma^2$の信頼区間水準0.95でそれぞれ区間推定せよ。
信頼水準0.95の$\mu$の信頼区間は
\[
\bar{x} \pm \frac{s}{\sqrt{n}} t_{a/2}(n-1)
\]
\[
\bar{x} = 62.5, n = 15,  =0.05,
t_{0.025}(14) = 2.1448 , s^2 = 110.25
\]
代入すると、(-56.69, 68.31)\\
従って、$\mu$の信頼区間は (-56.69, 68.31)

次に$\sigma^2$の信頼区間は
\[
(\frac{(n-1)s^2}{x_{a/2}^2(n-1)}, \frac{(n-1)s^2}{x_{1-{a/2}}^2(n-1)})
\]
\[
x^2_{0.025}(14) = 26.119, 
x^2_{0.975}(14) = 5.629
\]
代入すると、(59.09, 274,21)\\
従って,$\sigma^2$の信頼区間は(59.09, 274,21)

\section{}
帰無仮説H:$\mu$=57.5を有意水準0.05で両側検定せよ。\\
この実験では母分散が未知なので、不偏分散を用いる統計量tを使う。
\[
|t| = |\frac{\bar{x}-\mu_0}{\sqrt{v/n}}| >= t_{a/2}(n-1)
\]
\[
a=0.05, n=15, \bar{x}=62.5, s^2=110.25
\]
従って、
\[
|t| = |\frac{62.5-57.5}{\sqrt{110.25}/\sqrt{15}}| \approx 1.84427778391
\]
\[
t_{0.025}(14)= 2.1448 
\]
\[
|t| \approx 1.8442 < 2.1448 = t_{0.025}(14)
\]
ゆえに、有意水準0.05での両側検定の棄却域は、2.145以上または-2.145以下なので棄却されない。

\section{}
帰無仮説H:$\sigma^2$=120を有意水準0.1で両側検定せよ。\\
$\sigma$が既知なので、この標本平均値の検定量zを求める。また、zは標準分布に従う。
\[
z = \frac{v}{\sigma_{0}^2/n-1}
\]
標本不偏分散値を代入をすると、12.86になる。\\
自由度14の$\chi^2$分布の数表より、有意水準0.1での両側検定の棄却域は、\\
23.685以上または,6.571以下なので、棄却されない。
\end{document}