\documentclass[11pt]{jsarticle}
\usepackage{amsmath}
\begin{document}
レポート2\\
\begin{flushright}
  205713B\ 朝比奈太郎
\end{flushright}

pascal分布の確率関数は、次式で定義されている。\\
\[
P(X = k) = \frac{1}{1+\mu}(\frac{\mu}{1+\mu})^k \ \ \ \ \ \ \
(\mu > 0, k = 0,1,2, ...)
\]
確率変数であることを証明し、期待値と分散を求めよ。\\
\\
Answer\\
確率変数である条件は、 
\[
0 <= P(X) <= 1 かつ \sum P(X = k) = 1
\]
であることから、
\[
0 <= \frac{1}{1+\mu}(\frac{\mu}{1+\mu})^k <= 1
\]
となる。
また、
\[
p = \frac{1}{1+\mu}, \ q^{k-1} = (\frac{\mu}{1+\mu})^k
\]とすると
\[
\sum _{k=0}^{\infty} P(X = k) = \sum _{k=0}^{\infty} q^{k-1} p = p * \frac{1}{1-q} = 1
\]となる。
従って、与式は確率変数といえる。\\

確率分布が幾何分布に従うので、Xの期待値は、
\[
E(X) = \sum _{k=0}^{\infty}kP(X = k) = \sum _{k=0}^{\infty}kp^{k}(1-p) = (1-p)p* \sum _{k=0}^{\infty} k*p^{k-1} ...(1)
\]
である。右辺の総和を求めるために、関数 1/(1-p)のテーラ展開に着目すると、
\[
\frac{1}{1-p} = 1+p+p^2+... = \sum _{k=0}^{\infty}p^k
\]
左辺の微分は、
\[
(\frac{1}{1-p})' = \frac{1}{(1-p)^2}
\]
であり、右辺の微分は、
\[
(\sum _{k=0}^{\infty}p^k)' = \sum _{k=0}^{\infty}kp^{k-1}
\]
であるので、
\[
\frac{1}{(1-p)^2} = \sum _{k=0}^{\infty}kp^{k-1}
\]
となる。これを(1)に代入すると、
\[
E[X] = (1-p)p \frac{1}{(1-p)^2} = \frac{p}{1-p}
\]
また、分散は二乗期待値と期待値の二乗の差に等しいことから、
求める分散をV(X)とすると,E(X) = 1/pより
\[
V(X) = E[X^2] - \frac{1}{p^2}
\]
\[
E[X^2] =  \sum _{k=0}^{\infty}k^2(1-p)p^k =  (1-p)p\sum _{k=0}^{\infty}k^2p^{k-1}
\]
テイラー展開に対して、両辺をpについて2回微分する。
\[
\frac{2}{(1-p)^3} = \sum _{k=0}^{\infty}k(1-p)p^{k-2}
\]
両辺にpをかける。
\[
\frac{2p}{(1-p)^3} = \sum _{k=0}^{\infty}k(1-p)p^{k-1} =  \sum _{k=0}^{\infty}k^2p^{k-1} -  \sum _{k=0}^{\infty}kp^{k-1} =  \sum _{k=0}^{\infty}k^2p^{k-1} - \frac{1}{(1-p)^2}
\]
よって、
\[
E[X^2] = \frac{2p^2}{(1-p)^2} + \frac{p}{1-p}
V[X] = E[X^2] - E[X]^2 = \frac{2p^2}{(1-p)^2} + \frac{p}{1-p} - \frac{p^2}{(1-p)^2} = \frac{p}{(1-p)^2}
\]
\end{document}