\documentclass[11pt]{jsarticle}
\begin{document}
レポート1\\
\begin{flushright}
  205713B\ 朝比奈太郎
\end{flushright}

0-1\ で構成される3文字の文字集合が
\{000,\ 001,\ 010,\ 011,\ 100,\ 101,\ 110,\ 111\}で与えられたとき、最短超文字列を求めよ。\\
\\答え
\\"0001011100"\\
\\ 証明(訂正前)\\
まず、000と111はそれぞれ被っている部分がないので、必ず答えの一部になる。すると、取り敢えず"000111"となる。これは、000,001,011,111の4つを満たしている。残りは、010,100,101,110 である。"000111"の最後尾に00をつけると、 "00011100"となり、まだ満たしていない残りの要素は010,101のみとなる。"00011100"の前から3番目に10を追加すると、"0001011100"となり、これは010,101を満たす。従って、"0001011100"は最短超文字列といえる。\\

証明(訂正後)\\
長さ3の8部分列は全て異なっているので、各重なりは長くて2となる。
従って最短の長さは、$3\times 8-2\times7=10$となる。
次にパターンを発見するために、1つの部分から他の7つの部分列への重なりを数え、一覧をグラフとして準備する。このグラフから、重なりが大きいエッジを通るように部分列を順に並べれば良い。全てのエッジで重なりを2とできるので、長さ10の最短超文字列を生成できる。
従って、000,001,010,101,011,111,110,100の順で、 "00011011100"
\end{document}  